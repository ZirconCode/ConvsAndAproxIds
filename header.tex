%Author:
%------
%Yannis Baehni at University of Zurich
%baehni.yannis@uzh.ch
%adapted by simon gruening simon.gruening@uzh.ch
%
%Version log:
%-----------
%06/02/16 . Basic structure
%19/07/16 . Simon Happens

%Page Setup
\documentclass[
	10pt, 
	oneside, 
	a4paper, 
	footsepline
]{amsart}

\usepackage[
	left = 3cm, 
	right = 3cm, 
	top = 3cm, 
	bottom = 3cm
]{geometry}

%Manipulation of headers and footers
\usepackage{fancyhdr}
	\pagestyle{fancy}
	%Clear fields
	\fancyhf{}
	%Header
	\fancyhead[R]{
		\footnotesize
		Simon Gr\"uning\\
		\href{mailto:simon.gruening@uzh.ch}{simon.gruening@uzh.ch}
	}
	\fancyhead[L]{
		\footnotesize
		MAT694 Seminar: Introduction to Harmonic Analysis\\
		HS16
	}
	%Page numbering in footer
	\fancyfoot[C]{\thepage}
	%Separation line header and footer
	\renewcommand{\headrulewidth}{0.4pt}
	\renewcommand{\footrulewidth}{0.4pt}

%Title
\title{\Huge Classical Fourier Analysis: Convolution and Approximate Identities}

\advance\footskip0.4cm
\textheight=54pc    %a4paper
%\textheight=50.5pc %letterpaper
\advance\textheight-0.4cm
\calclayout

%Font settings
\usepackage{anyfontsize}

%Footnote settings
\usepackage{footmisc}
%	\renewcommand*{\thefootnote}{\fnsymbol{footnote}}

%Further math environments
\usepackage{amsmath}
%Further math fonts (loads amsfonts implicitely)
\usepackage{amssymb}
%Redefinition of \text
\usepackage{amstext}
%Polynomial division
\usepackage{polynom}


%Graphics
\usepackage{graphicx}
\usepackage{caption}
\usepackage{subcaption}
\usepackage{afterpage}
%Frames
\usepackage{mdframed}
\makeatletter
\ifcase \@ptsize \relax% 10pt
  \newcommand{\miniscule}{\@setfontsize\miniscule{5}{6}}% \tiny: 5/6
\fi
\makeatother
 
\usepackage{hhline}
\usepackage{booktabs} 
\usepackage{array}
\usepackage{xfrac} 
\DeclareMathSizes{10}{10}{7}{6}
\makeatletter
\renewcommand*\env@matrix[1][*\c@MaxMatrixCols c]{%
  \hskip -\arraycolsep
  \let\@ifnextchar\new@ifnextchar
  \array{#1}}
\makeatother
%\everymath{\displaystyle}
\newcommand{\Space}{\vspace{2mm}}
%Enumerate
\usepackage{enumitem} 
\renewcommand{\labelitemi}{$\bullet$}
\renewcommand{\labelitemii}{$\ast$}
\renewcommand{\labelitemiii}{$\cdot$}
\renewcommand{\labelitemiv}{$\circ$}
\newcommand*\circled[1]{\tikz[baseline=(char.base)]{\node[shape=circle,draw,inner sep = 2pt](char){#1};}}
%Colors
\usepackage{color}
\usepackage[cmtip, all]{xy}
%Theorems
\usepackage{amsthm}
\newtheoremstyle{bold}              	 % Name
  {}                                     % Space above
  {}                                     % Space below
  {\itshape}                             % Body font
  {}                                     % Indent amount
  {\bfseries}                            % Theorem head font
  {.}                                    % Punctuation after theorem head
  { }                                    % Space after theorem head, ' ', or \newline
  {} 
\theoremstyle{bold}
\newtheorem{definition}{Definition}[section]
\newtheorem{lemma}{Lemma}[section]
\newtheorem{Proof}{Proof}[section]
\newtheorem{proposition}{Proposition}[section]
\newtheorem{properties}{Properties}[section]
\newtheorem{corollary}{Corollary}[section]
\newtheorem{theorem}{Theorem}[section]
\newtheorem{example}{Example}[section]
\newtheorem{remark}{Remark}[section]
%German non-ASCII-Characters
\usepackage[utf8]{inputenc}
%Graphics-Tool
\usepackage{tikz}
\usepackage{tikzscale}
\usepackage{bbm}
\usetikzlibrary{calc,fadings,decorations.pathreplacing}
\usepackage{verbatim}
\usepackage{amssymb}
\usepackage{array}
\newcolumntype{C}[1]{>{\centering\arraybackslash}p{#1}}
\usetikzlibrary{arrows, matrix}
%\usepackage{bera}
%Listing-Setup
\usepackage{relsize}
\usepackage{color}
\definecolor{gray}{gray}{0.55}
%Titlepage Command
\newcommand{\HRule}{\rule{\linewidth}{0.5mm}}
\setcounter{section}{-1}
%Bibliographie
\usepackage[backend=bibtex, style=alphabetic]{biblatex}
%\usepackage[babel, german = swiss]{csquotes}
\bibliography{Bibliography}
%PDF-Linking
\usepackage[bookmarksopen=true,bookmarksnumbered=true]{hyperref}

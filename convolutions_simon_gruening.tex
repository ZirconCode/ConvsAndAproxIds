%Author:
%------
%Yannis Baehni at University of Zurich
%baehni.yannis@uzh.ch
%adapted by simon gruening simon.gruening@uzh.ch
%
%Version log:
%-----------
%06/02/16 . Basic structure
%19/07/16 . Simon Happens

%Page Setup
\documentclass[
	10pt, 
	oneside, 
	a4paper, 
	footsepline
]{amsart}

\usepackage[
	left = 3cm, 
	right = 3cm, 
	top = 3cm, 
	bottom = 3cm
]{geometry}

%Manipulation of headers and footers
\usepackage{fancyhdr}
	\pagestyle{fancy}
	%Clear fields
	\fancyhf{}
	%Header
	\fancyhead[R]{
		\footnotesize
		Simon Gr\"uning\\
		\href{mailto:simon.gruening@uzh.ch}{simon.gruening@uzh.ch}
	}
	\fancyhead[L]{
		\footnotesize
		MAT694 Seminar: Introduction to Harmonic Analysis\\
		HS16
	}
	%Page numbering in footer
	\fancyfoot[C]{\thepage}
	%Separation line header and footer
	\renewcommand{\headrulewidth}{0.4pt}
	\renewcommand{\footrulewidth}{0.4pt}

%Title
\title{\Huge Classical Fourier Analysis: Convolution and Approximate Identities}

\advance\footskip0.4cm
\textheight=54pc    %a4paper
%\textheight=50.5pc %letterpaper
\advance\textheight-0.4cm
\calclayout

%Font settings
\usepackage{anyfontsize}

%Footnote settings
\usepackage{footmisc}
%	\renewcommand*{\thefootnote}{\fnsymbol{footnote}}

%Further math environments
\usepackage{amsmath}
%Further math fonts (loads amsfonts implicitely)
\usepackage{amssymb}
%Redefinition of \text
\usepackage{amstext}
%Polynomial division
\usepackage{polynom}


%Graphics
\usepackage{graphicx}
\usepackage{caption}
\usepackage{subcaption}
\usepackage{afterpage}
%Frames
\usepackage{mdframed}
\makeatletter
\ifcase \@ptsize \relax% 10pt
  \newcommand{\miniscule}{\@setfontsize\miniscule{5}{6}}% \tiny: 5/6
\fi
\makeatother
 
\usepackage{hhline}
\usepackage{booktabs} 
\usepackage{array}
\usepackage{xfrac} 
\DeclareMathSizes{10}{10}{7}{6}
\makeatletter
\renewcommand*\env@matrix[1][*\c@MaxMatrixCols c]{%
  \hskip -\arraycolsep
  \let\@ifnextchar\new@ifnextchar
  \array{#1}}
\makeatother
%\everymath{\displaystyle}
\newcommand{\Space}{\vspace{2mm}}
%Enumerate
\usepackage{enumitem} 
\renewcommand{\labelitemi}{$\bullet$}
\renewcommand{\labelitemii}{$\ast$}
\renewcommand{\labelitemiii}{$\cdot$}
\renewcommand{\labelitemiv}{$\circ$}
\newcommand*\circled[1]{\tikz[baseline=(char.base)]{\node[shape=circle,draw,inner sep = 2pt](char){#1};}}
%Colors
\usepackage{color}
\usepackage[cmtip, all]{xy}
%Theorems
\usepackage{amsthm}
\newtheoremstyle{bold}              	 % Name
  {}                                     % Space above
  {}                                     % Space below
  {\itshape}                             % Body font
  {}                                     % Indent amount
  {\bfseries}                            % Theorem head font
  {.}                                    % Punctuation after theorem head
  { }                                    % Space after theorem head, ' ', or \newline
  {} 
\theoremstyle{bold}
\newtheorem{definition}{Definition}[section]
\newtheorem{lemma}{Lemma}[section]
\newtheorem{Proof}{Proof}[section]
\newtheorem{proposition}{Proposition}[section]
\newtheorem{properties}{Properties}[section]
\newtheorem{corollary}{Corollary}[section]
\newtheorem{theorem}{Theorem}[section]
\newtheorem{example}{Example}[section]
\newtheorem{remark}{Remark}[section]
%German non-ASCII-Characters
\usepackage[utf8]{inputenc}
%Graphics-Tool
\usepackage{tikz}
\usepackage{tikzscale}
\usepackage{bbm}
\usetikzlibrary{calc,fadings,decorations.pathreplacing}
\usepackage{verbatim}
\usepackage{amssymb}
\usepackage{array}
\newcolumntype{C}[1]{>{\centering\arraybackslash}p{#1}}
\usetikzlibrary{arrows, matrix}
%\usepackage{bera}
%Listing-Setup
\usepackage{relsize}
\usepackage{color}
\definecolor{gray}{gray}{0.55}
%Titlepage Command
\newcommand{\HRule}{\rule{\linewidth}{0.5mm}}
\setcounter{section}{-1}
%Bibliographie
\usepackage[backend=bibtex, style=alphabetic]{biblatex}
%\usepackage[babel, german = swiss]{csquotes}
\bibliography{Bibliography}
%PDF-Linking
\usepackage[bookmarksopen=true,bookmarksnumbered=true]{hyperref}

\begin{document}

\title{Convolution and Approximate Identities}
\author{Simon Gr\"uning}
\address[Simon Gr\"uning]{University of Zurich, R\"{a}mistrasse 71, 8006 Zurich}
\email[Simon Gr\"uning]{\href{mailto:simon.gruening@uzh.ch}{simon.gruening@uzh.ch}}



\maketitle
\addtocounter{section}{1}

\section{Examples of Topological Groups}



\begin{definition}
Topological Group
\end{definition}

\begin{definition}
Locally Compact
\end{definition}

\begin{definition}
Haar Measure
\end{definition}

	\begin{example}
	$\mathbb{R}^n, \mathbb{Z}^n, \mathbb{T}^n$
	\end{example}
	
\begin{example}
$dx/|x|$
\end{example}	
	
\begin{example}
Heisenberg Group $\mathbb{H}^n$
\end{example}	

	
\section{Convolution}

\begin{definition}
 Let $f,g \in L^1(G)$. Define the \textbf{convolution} $f*g$ by
\begin{equation}
 (f*g)(x) := \int_G f(y)g(y^{-1}x) d\lambda(y)
\end{equation}
\end{definition}

Note that on $\mathbb{R}^n$ with an additive structure (our preferred environment for later chapters), we will simply have:

\begin{equation}
 (f*g)(x) = \int_{\mathbb{R}^n} f(y)g(x-y) dy
\end{equation}

\begin{theorem}
Convolution is define $\lambda$ almost everywhere. To see this we take the $L_1$ norm on the definition to find it finite:

\begin{align*}
\tag{Apply Norm}
 \int_G \int_G \abs{f(y)} \abs{g(y^{-1}x)} d\lambda(y) d\lambda(x) = \\
 \tag{Fubini}
 \int_G \int_G \abs{f(y)} \abs{g(y^{-1}x)} d\lambda(x) d\lambda(y) \\
 \tag{Measure-Invariance}
 \int_G \abs{f(y)} \int_G  \abs{g(y^{-1}x)} d\lambda(x) d\lambda(y) \\
 \tag{Left Haar}
 \int_G \abs{f(y)} \int_G  \abs{g(x)} d\lambda(x) d\lambda(y) \\
 \tag{Definition}
 blah \\
\end{align*}

\end{theorem}

\section{Basic Convolution Inequalities}


\begin{theorem}
Minkowskis Inequality, triangle inequality for Lp spaces
\end{theorem}

\begin{theorem}
Youngs Inequality
\end{theorem}


\begin{theorem}
Youngs Inequality for Weak Type Spaces
ouch proof
\end{theorem}

\section{Approximate Identities}

Approximation of dirac delta function , identity element of convolutions

\begin{definition}
An \textbf{approximate identity} (as $\varepsilon \rightarrow 0$) is a family of $L^1(G)$ functions $k_\varepsilon$ with the following three properties: 
\begin{enumerate}[label=(\roman*)]
\item There exists a constant $c > 0$ such that $\| k_\varepsilon \| _ {L^1(G)} \leqslant c$ for all $ \varepsilon > 0 $ . 
\item $ \int_G k_\varepsilon(x) d\lambda(x) = 1$ for all $\varepsilon > 0 $.
\item For any neighborhood $V$ of the identity element $e$ of the group G we have $ \int_{V^c} | k_\varepsilon(x) | d \lambda(x) \rightarrow 0  $ as $ \varepsilon \rightarrow 0 $ . 
\end{enumerate}
\end{definition}

\begin{example}
\begin{figure}[h!bt]
\centering
\includegraphics[width = \textwidth]{matlab/fejerkernel}
\caption{Fejer Kernel}
\label{fejer}
\end{figure}
\end{example}

\begin{theorem}
approx. id. on locallz compact group G with left Haar measure
\end{theorem}

\begin{theorem}
ke familz of funcs on loc compact group G with properties...
\end{theorem}

\section{Required Stuff}

\begin{enumerate}
\item hausdorf topological space
\item counting measure
\item area of intersecting circles
\item banach algebra
\item hoelders inequality
\item fubini
\item chebyschevs inequality
\item lebesgue dominated conv. thm.
\item measure theoretic support
\end{enumerate}

chapter 1 stuff:
\begin{enumerate}
\item Lp norms and other defs etc.
\item distr. functions
\end{enumerate}

\end{document}
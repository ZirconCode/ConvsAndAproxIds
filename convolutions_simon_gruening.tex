%Author:
%------
%Yannis Baehni at University of Zurich
%baehni.yannis@uzh.ch
%adapted by simon gruening simon.gruening@uzh.ch
%
%Version log:
%-----------
%06/02/16 . Basic structure
%19/07/16 . Simon Happens

%Page Setup
\documentclass[
	10pt, 
	oneside, 
	a4paper, 
	footsepline
]{amsart}

\usepackage[
	left = 3cm, 
	right = 3cm, 
	top = 3cm, 
	bottom = 3cm
]{geometry}

%Manipulation of headers and footers
\usepackage{fancyhdr}
	\pagestyle{fancy}
	%Clear fields
	\fancyhf{}
	%Header
	\fancyhead[R]{
		\footnotesize
		Simon Gr\"uning\\
		\href{mailto:simon.gruening@uzh.ch}{simon.gruening@uzh.ch}
	}
	\fancyhead[L]{
		\footnotesize
		MAT694 Seminar: Introduction to Harmonic Analysis\\
		HS16
	}
	%Page numbering in footer
	\fancyfoot[C]{\thepage}
	%Separation line header and footer
	\renewcommand{\headrulewidth}{0.4pt}
	\renewcommand{\footrulewidth}{0.4pt}

%Title
\title{\Huge Classical Fourier Analysis: Convolution and Approximate Identities}

\advance\footskip0.4cm
\textheight=54pc    %a4paper
%\textheight=50.5pc %letterpaper
\advance\textheight-0.4cm
\calclayout

%Font settings
\usepackage{anyfontsize}

%Footnote settings
\usepackage{footmisc}
%	\renewcommand*{\thefootnote}{\fnsymbol{footnote}}

%Further math environments
\usepackage{amsmath}
%Further math fonts (loads amsfonts implicitely)
\usepackage{amssymb}
%Redefinition of \text
\usepackage{amstext}
%Polynomial division
\usepackage{polynom}


%Graphics
\usepackage{graphicx}
\usepackage{caption}
\usepackage{subcaption}
\usepackage{afterpage}
%Frames
\usepackage{mdframed}
\makeatletter
\ifcase \@ptsize \relax% 10pt
  \newcommand{\miniscule}{\@setfontsize\miniscule{5}{6}}% \tiny: 5/6
\fi
\makeatother
 
\usepackage{hhline}
\usepackage{booktabs} 
\usepackage{array}
\usepackage{xfrac} 
\DeclareMathSizes{10}{10}{7}{6}
\makeatletter
\renewcommand*\env@matrix[1][*\c@MaxMatrixCols c]{%
  \hskip -\arraycolsep
  \let\@ifnextchar\new@ifnextchar
  \array{#1}}
\makeatother
%\everymath{\displaystyle}
\newcommand{\Space}{\vspace{2mm}}
%Enumerate
\usepackage{enumitem} 
\renewcommand{\labelitemi}{$\bullet$}
\renewcommand{\labelitemii}{$\ast$}
\renewcommand{\labelitemiii}{$\cdot$}
\renewcommand{\labelitemiv}{$\circ$}
\newcommand*\circled[1]{\tikz[baseline=(char.base)]{\node[shape=circle,draw,inner sep = 2pt](char){#1};}}
%Colors
\usepackage{color}
\usepackage[cmtip, all]{xy}
%Theorems
\usepackage{amsthm}
\newtheoremstyle{bold}              	 % Name
  {}                                     % Space above
  {}                                     % Space below
  {\itshape}                             % Body font
  {}                                     % Indent amount
  {\bfseries}                            % Theorem head font
  {.}                                    % Punctuation after theorem head
  { }                                    % Space after theorem head, ' ', or \newline
  {} 
\theoremstyle{bold}
\newtheorem{definition}{Definition}[section]
\newtheorem{lemma}{Lemma}[section]
\newtheorem{Proof}{Proof}[section]
\newtheorem{proposition}{Proposition}[section]
\newtheorem{properties}{Properties}[section]
\newtheorem{corollary}{Corollary}[section]
\newtheorem{theorem}{Theorem}[section]
\newtheorem{example}{Example}[section]
\newtheorem{remark}{Remark}[section]
%German non-ASCII-Characters
\usepackage[utf8]{inputenc}
%Graphics-Tool
\usepackage{tikz}
\usepackage{tikzscale}
\usepackage{bbm}
\usetikzlibrary{calc,fadings,decorations.pathreplacing}
\usepackage{verbatim}
\usepackage{amssymb}
\usepackage{array}
\newcolumntype{C}[1]{>{\centering\arraybackslash}p{#1}}
\usetikzlibrary{arrows, matrix}
%\usepackage{bera}
%Listing-Setup
\usepackage{relsize}
\usepackage{color}
\definecolor{gray}{gray}{0.55}
%Titlepage Command
\newcommand{\HRule}{\rule{\linewidth}{0.5mm}}
\setcounter{section}{-1}
%Bibliographie
\usepackage[backend=bibtex, style=alphabetic]{biblatex}
%\usepackage[babel, german = swiss]{csquotes}
\bibliography{Bibliography}
%PDF-Linking
\usepackage[bookmarksopen=true,bookmarksnumbered=true]{hyperref}

\begin{document}

\title{Convolution and Approximate Identities}
\author{Simon Gr\"uning}
\address[Simon Gr\"uning]{University of Zurich, R\"{a}mistrasse 71, 8006 Zurich}
\email[Simon Gr\"uning]{\href{mailto:simon.gruening@uzh.ch}{simon.gruening@uzh.ch}}

\maketitle


\section{Setting the Scene}

\begin{definition}
$G$ is a Topological Group if and only if the following are true:
\begin{enumerate}
\item $G$ is a topological space
\item $G$ is Hausdorff
\item $G$ satisfies the group axioms
\item The maps $ (x,y) \mapsto xy $ and $ x \mapsto x^{-1}$ are continuous
\end{enumerate}
\end{definition}

\begin{definition}
A Topological Group $G$ is Locally Compact if 
\begin{equation*}
\exists U \subset G \text{ open}: e \in U \wedge \overline{U} \text{is compact}
\end{equation*}
where $e$ is the identity element of $G$.
\end{definition}

In the remaining text $G$ shall be a Locally Compact Topological Group. It is not necessary for $G$ to be abelian in most cases, however we shall assume so. Where it is relevant, we have included footnotes. We shall also make use of a specific measure on G:

\begin{lemma}
There exists a measure $\lambda$ on any topological group $G$ defined on the Borel sets such that:
\begin{enumerate}
\item $U \subset G \wedge U \neq \emptyset \wedge U \text{ open} \implies \lambda(U) > 0$
\item $U \subset G \wedge U \text{ compact} \implies \lambda(U) < \infty$
\item $\lambda$ is unique up to multiplication
\item $\forall t \in G: A \subset G \text{ measurable} \implies \lambda(tA) = \lambda(A)$
\end{enumerate}
This measure is called the Left Haar Measure, highlighted by the final property of left invariance.
\end{lemma}

\begin{proof}
For conclusive proof see the Appendix in Grafakous.
\end{proof}

In the remaining text $\lambda$ shall denote the Left Haar Measure on $G$. We shall make one final assumption, that $G$ is a countable union of compact subsets. This makes $(G,\lambda)$ a $\sigma$-finite measure space. We now have that which is necessary to begin.
	
\section{Convolution}

\begin{definition}
 Let $f,g \in L^1(G)$. Define the \textbf{convolution} $f*g$ by
\begin{equation}
 (f*g)(x) := \int_G f(y)g(y^{-1}x) d\lambda(y)
\end{equation}
\end{definition}

\begin{remark}
Note that on $\mathbb{R}^n$ with an additive structure (our preferred environment for later chapters), we will simply have:
\begin{equation*}
 (f*g)(x) = \int_{\mathbb{R}^n} f(y)g(x-y) dy
\end{equation*}
\end{remark}

\begin{example}
Let $G = \mathbb{R}$ with the usual additive structure and 
\begin{equation}
f(x) = \begin{cases}
1 & -1 \leq x \leq 1 \\
0 & \text{else}
\end{cases}
\end{equation}
Then we calculate:
\begin{align*}
(f \ast f)(x) &= \int_\mathbb{R} f(y)f(x-y) d\lambda(y) \\
&= 
\begin{cases}
\displaystyle
\int_\mathbb{R} \chi_{\intcc{-1,1}\cap\intcc{x-1,x+1}}(x) d\lambda(x) & -2 \leq x \leq 2 \\
\displaystyle
\int_\mathbb{R} 0 d\lambda(y) & \text{else}
\end{cases} \\
&= 
\begin{cases}
\displaystyle
2-\abs{x} & -2 \leq x \leq 2 \\
\displaystyle
0 & \text{else}
\end{cases} 
\end{align*}

Notice that the convolution operator has a natural smoothing effect on $f$, as it does on every function.

\end{example}

\begin{lemma}
Convolution is defined $\lambda$ almost everywhere. 
\end{lemma}

\begin{proof}To see this we take the $L_1$ norm on the definition to find it finite:
\begin{align*}
 \norm{(f \ast g)}_{L^1} &= \int_G \abs{ \int_G f(y) g(y^{-1}x) d\lambda(y) } d\lambda(x) 
\\ 
&\leq \int_G \int_G \abs[0]{f(y)} \abs[0]{g(y^{-1}x)} d\lambda(y) d\lambda(x) \\
&=  \int_G \int_G \abs[0]{f(y)} \abs[0]{g(y^{-1}x)} d\lambda(x) d\lambda(y) \\
&=  \int_G \abs[0]{f(y)} \int_G  \abs[0]{g(y^{-1}x)} d\lambda(x) d\lambda(y) \\
&=  \int_G \abs[0]{f(y)} \int_G  \abs[0]{g(x)} d\lambda(x) d\lambda(y) \\
 &= \norm{f}_{L^1} \norm{g}_{L^1} \\
 &< \infty
\end{align*}
This proof followed simply from the triangle inequality for integrals, the definition of our norm, Fubini, and the properties of our measure.
\end{proof}

\begin{lemma}
\begin{equation*}
(f \ast g)(x) = \int_G f(xz)g(z^{-1}) d\lambda(z)
\end{equation*}
\end{lemma}

\begin{proof}
We perform a change of variables $z = x^{-1}y$:
\begin{align*}
(f \ast g)(x) &= \int_G f(y)g(y^{-1}x) d\lambda(y) \\
&= \int_G f(xx^{-1}y)g((yx^{-1})^{-1}) d\lambda(y) \\
&= \int_G f(xz)g(z^{-1}) d\lambda(x^{-1}y) \\
&= \int_G f(xz)g(z^{-1}) d\lambda(z) \\
\end{align*}
Where we have made use of the inherent property of a Left Haar Measure.
\end{proof}

\begin{proposition}

$\forall f,g,h \in L^1(G):$
\begin{enumerate}
\item
$f \ast (g \ast h) = (f \ast g) \ast h $
\item
$f \ast (g + h) = f \ast g + f \ast h \wedge (f+g) \ast h = f\ast h + f\ast g $
\end{enumerate}
Thus convolution is associative and distributive.
\end{proposition}

\begin{proof}
Associativity, as so many things, follows from an application of Fubini:
\begin{align*}
((f \ast g) \ast  h )(x) &= \int_G (f \ast g)(y) h(y^{-1}x) d\lambda(y) \\
&= \int_G ( \int_G f(z)g(z^{-1}y)d\lambda(z)) h(y^{-1}x)d\lambda(y) \\
&= \int_G  \int_G f(z)g(z^{-1}y) h(y^{-1}x)d\lambda(z) d\lambda(y) \\
&= \int_G  f(z) \int_G g(z^{-1}y) h(y^{-1}x)d\lambda(y) d\lambda(z) \\
&= \int_G  f(z) (g \ast h)(xz^{-1}) d\lambda(z) \\
&= (f \ast (g \ast  h ))(x) \\
\end{align*}
Distributivity follow from the additive property of the integral:
\begin{align*}
f \ast (g + h) &= \int_G f(y) (g+h)(y^{-1}x)d\lambda(y)\\
&= \int_G f(y) (g(y^{-1}x)+h(y^{-1}x))d\lambda(y)\\
&= \int_G f(y)g(y^{-1}x)+f(y)h(y^{-1}x)d\lambda(y)\\
&= \int_G f(y)g(y^{-1}x) d\lambda(y) + \int_G f(y)h(y^{-1}x) d\lambda(y)\\
&= f \ast g + f \ast h
\end{align*}
The mirror statement follows analogously. 
\end{proof}

\section{Basic Convolution Inequalities}

\begin{definition}
Define $ p' := p / (p-1) $. To maintain our desired property in infinity we also declare: $1 / \infty = 0$
\end{definition}

\begin{remark}
Notice then $\frac{1}{p} + \frac{1}{p'} = \frac{1}{p} + \frac{p-1}{p} = \frac{p}{p} = 1$ for any $p \in (0,\infty]$
\end{remark}

\begin{remark}
In the following proofs we often make use of this useful inequality without further explication, it allows us to manipulate inequalities without worrying about negatives:
\begin{align*}
f \ast g &= \int_G f(y)g(y^{-1}x) d\lambda(y) \\
&\leq \abs{ \int_G f(y)g(y^{-1}x) d\lambda(y) } \\
&\leq \int_G \abs[0]{f(y)}\abs[0]{g(y^{-1}x)} d\lambda(y) \\
&= \abs{f} \ast \abs{g}
\end{align*}
\end{remark}

What follows is akin to the triangle inequality for $L^p$ spaces.

\begin{theorem}
Minkowskis Inequality: Let $1 \leq p \leq \infty$, $f \in L^p (G)$, $g \in L^1(G)$ then it follows that: 
$g \ast f $exists $\lambda$-almost-everywhere and
$ \norm{g \ast f}_{L^p(G)} \leq \norm{g}_{L^1(G)} \norm{f}_{L^p(G)} $
\end{theorem}

\begin{proof}
First we notice that the case of $p=1$ follows from (Lemma 3.1). Similarly\footnote{This only works with the Abelian property, however it is possible to assume some small additional necessities to make it work in non-commutative groups} we may rid ourselves of the case $p=\infty$ by again applying the triangle inequality and recalling that $\abs{f} \leq \norm{f}_{L^\infty}$ $\lambda$-almost-everywhere:
\begin{align*}
\abs{(f \ast g)(x)} &= \abs{\int_G f(y)g(y^{-1}x) d\lambda(y)} \\
&\leq \int_G \abs{f(y)} \abs{g(y^{-1}x)} d\lambda(y) \\
&\leq \norm{f}_{L^\infty} \int_G \abs{g(y^{-1}x)} d\lambda(y) \\
&= \norm{f}_{L^\infty} \int_G \abs{g(y)} d\lambda(y) \\
&= \norm{f}_{L^\infty} \norm{g}_{L^1}
\end{align*}

For $ 1 < p < \infty$, we must work a little harder. We have:
\begin{equation*}
(\abs{g} \ast \abs{f} )(x) = \int_G \abs{f(y^{-1}x)}\abs{g(y)} d\lambda(y)
\end{equation*}
We shall apply H\"olders inequality with the to following criteria,

\begin{description}
[align=right,labelwidth=3cm]
\item [Measure] $\abs{g(y)} d\lambda(y)$
\item [Functions] $1$ and $\abs{f(y^{-1}x)}$
\item [Exponents] $p$ and $p'$ since $\frac{1}{p}+\frac{1}{p'} = 1$
\end{description}

to discover:

\begin{align*}
(\abs{g} \ast \abs{f} )(x) &\leq (\int_G \abs{f(y^{-1}x)}^p \abs{g(y)} d\lambda(y))^{1/p} (\int_G 1^{p'} \abs{g(y)} 
 d\lambda(y))^{1/p'} \\
 &= (\int_G \abs{f(y^{-1}x)}^p \abs{g(y)} d\lambda(y))^{1/p} (\int_G \abs{g(y)} 
 d\lambda(y))^{1/p'}
\end{align*}

Since we are working with positive functions, by the monotonicity of the exponent and the integral, we may take the $L^p$ norm on both sides while preserving the inequality.

\begin{align*}
\norm{\abs{g} \ast \abs{f}}_{L^p} &\leq
\norm{(\int_G \abs{g(y)} 
 d\lambda(y))^{1/p'}(\int_G \abs{f(y^{-1}x)}^p \abs{g(y)} d\lambda(y))^{1/p} }_{L^p} \\
&= ((\int_G \abs{g(y)} d\lambda(y))^\frac{p}{p'} \int_G \int_G \abs{f(y^{-1}x)}^p \abs{g(y)} d\lambda(y)d\lambda(x) )^\frac{1}{p} \\
&= (\norm{g}^{p-1}_{L^1} \int_G \int_G \abs{f(y^{-1}x)}^p \abs{g(y)} d\lambda(y)d\lambda(x) )^\frac{1}{p} \\
&= (\norm{g}^{p-1}_{L^1} \int_G \int_G \abs{f(y^{-1}x)}^p \abs{g(y)} d\lambda(x)d\lambda(y) )^\frac{1}{p} \\
&= (\norm{g}^{p-1}_{L^1} \int_G \int_G \abs{f(y^{-1}x)}^p  d\lambda(x) \abs{g(y)} d\lambda(y) )^\frac{1}{p} \\
&= (\norm{g}^{p-1}_{L^1} \int_G \int_G \abs{f(x)}^p  d\lambda(x) \abs{g(y)} d\lambda(y) )^\frac{1}{p} \\
&= (\norm{g}^{p-1}_{L^1} \int_G \abs{f(x)}^p  d\lambda(x) \int_G  \abs{g(y)} d\lambda(y) )^\frac{1}{p} \\
&= (\norm{g}^{p-1}_{L^1} \norm{f}^p_{L^p} \norm{g}_{L^1})^\frac{1}{p} \\
&= (\norm{g}^{p}_{L^1} \norm{f}^p_{L^p})^\frac{1}{p} \\
&= \norm{g}_{L^1} \norm{f}_{L^p}\\
&< \infty
\end{align*}

This implies that $\abs{g}\ast\abs{f}$ is finite $\lambda$-a.e. and satisfies the required inequality. We have shown before that $ g \ast f \leq \abs{g}\ast\abs{f}$ which concludes our proof.

\end{proof}

We shall find that we can prove this same inequality in a more general case, strengthening our tool set by a lot.

\begin{theorem}
Young's Inequality: Let $1 \leq p,q,r \leq \infty$ such that
\begin{equation*}
\frac{1}{q} + 1 = \frac{1}{p}+\frac{1}{r}
\end{equation*}
and $f \in L^p \wedge g\in L^r$. Then $f \ast g$ exists $\lambda$-a.e and the following inequality holds:
\begin{equation*}
\norm{f \ast g}_{L^q} \leq \norm{g}_{L^r}\norm{f}_{L^p}
\end{equation*}
\end{theorem}

\begin{proof}
First we shall rid ourselves of the case $r = \infty$. We notice that then $ p = 1 \wedge q = \infty $ follows by satisfaction of the requirements. This case then simply follows from Minkowski's Inequality.

Therefore let us now assume $ r < \infty$. We examine first the beautiful relationship between $p,q,r$:

\begin{enumerate}
\item $1+\frac{1}{q} = \frac{1}{r}+\frac{1}{p} \Leftrightarrow 1-\frac{1}{r}+\frac{1}{q}+1-\frac{1}{p} = 1 \Leftrightarrow \frac{r-1}{r}+\frac{1}{q}+\frac{p-1}{p}=1 \Leftrightarrow \frac{1}{r/(r-1)}+\frac{1}{q}+\frac{1}{p/(p-1)}=1 \Leftrightarrow \frac{1}{r'}+\frac{1}{q}+\frac{1}{p'} =1$
\item $ \frac{1}{q}+1=\frac{1}{r}+\frac{1}{p} \Leftrightarrow \frac{p}{q}+p=\frac{p}{r}+1 \Leftrightarrow \frac{p}{q}+p-\frac{p}{r}=1 \Leftrightarrow \frac{p}{q}+\frac{p(r-1)}{r}=1 \Leftrightarrow \frac{p}{q}+\frac{p}{r/(r-1)}=1 \Leftrightarrow \frac{p}{q}+\frac{p}{r'}=1$
\item $ \frac{1}{q}+1 = \frac{1}{p}+\frac{1}{r} \Leftrightarrow \frac{1}{q}+1-\frac{1}{p}=\frac{1}{r} \Leftrightarrow \frac{r}{q}+r-\frac{r}{p}=1 \Leftrightarrow \frac{r}{q}+\frac{r(p-1)}{p}=1 \Leftrightarrow \frac{r}{q}+\frac{r}{p/(p-1)}=1 \Leftrightarrow \frac{r}{q}+\frac{r}{p'}=1$
\end{enumerate}
We shall make use of these lovely final 2 statements in what follows to split our terms into more suitable pieces. We work with $\abs{f} \ast \abs{g}$ as before:

\begin{align*}
\abs{(\abs{f}\ast \abs{g})(x)} &\leq \int_G \abs{f(y)}\abs{g(y^{-1}x)} d\lambda(y) \\
&= \int_G \abs{f(y)}^\frac{p}{r'}\abs{f(y)}^\frac{p}{q}\abs{g(y^{-1}x)}^\frac{r}{q}\abs{g(y^{-1}x)}^\frac{r}{p'} d\lambda(y) \\
&= \norm{f(y)^\frac{p}{r'} f(y)^\frac{p}{q} g(y^{-1}x)^\frac{r}{q}g(y^{-1}x)^\frac{r}{p'}}_{L^1}
\end{align*}

To proceed we shall now make use of H\"older's inducted inequality (Exercise 1.1.2(a)) for multiple exponents  with:
\begin{description}
[align=right,labelwidth=3cm]
\item [Measure] Simply $d\lambda(y)$
\item [Functions] $\abs{f(y)}^\frac{p}{r'}$, $\abs{g(y^{-1}x}^\frac{r}{p'}$, and by combining the rest, $\abs{f(y)}^\frac{p}{q}\abs{g(y^{-1}x)}^\frac{r}{q}$
\item [Exponents] $r'$, $q$, and $p'$. See our first discovery above.
\end{description}


To receive:

\begin{align*}
\norm{f(y)^\frac{p}{r'} f(y)^\frac{p}{q} g(y^{-1}x)^\frac{r}{q}g(y^{-1}x)^\frac{r}{p'}}_{L^1} &\leq \norm{f(y)^\frac{p}{r'}}_{L^{r'}} \norm{f(y)^\frac{p}{q} g(y^{-1}x)^\frac{r}{q}}_{L^q} \norm{g(y^{-1}x)^\frac{r}{p'}}_{L^{p'}} \\
&= (\int_G \abs{f(y)^\frac{p}{r'}}^{r'} d\lambda(y))^\frac{1}{r'} \norm{f(y)^\frac{p}{q} g(y^{-1}x)^\frac{r}{q}}_{L^q} \norm{g(y^{-1}x)^\frac{r}{p'}}_{L^{p'}} \\
&= (\int_G \abs{f(y)^p} d\lambda(y))^\frac{1}{r'} \norm{f(y)^\frac{p}{q} g(y^{-1}x)^\frac{r}{q}}_{L^q} \norm{g(y^{-1}x)^\frac{r}{p'}}_{L^{p'}} \\
&= \norm{f}^\frac{p}{r'}_{L^p} \norm{f(y)^\frac{p}{q} g(y^{-1}x)^\frac{r}{q}}_{L^q} \norm{g(y^{-1}x)^\frac{r}{p'}}_{L^{p'}} \\
&= \norm{f}^\frac{p}{r'}_{L^p} \norm{f(y)^\frac{p}{q} g(y^{-1}x)^\frac{r}{q}}_{L^q} (\int_G ( \abs{g(y^{-1}x)}^\frac{r}{p'})^{p'} d\lambda(y) )^\frac{1}{p'} \\
&= \norm{f}^\frac{p}{r'}_{L^p} \norm{f(y)^\frac{p}{q} g(y^{-1}x)^\frac{r}{q}}_{L^q} (\int_G  \abs{g(y^{-1}x)}^r d\lambda(y) )^\frac{1}{p'} \\
&= \norm{f}^\frac{p}{r'}_{L^p} (\int_G (\abs{f(y)^\frac{p}{q} g(y^{-1}x)^\frac{r}{q}} )^q)^\frac{1}{q} (\int_G  \abs{g(y^{-1}x)}^r d\lambda(y) )^\frac{1}{p'} \\
&= \norm{f}^\frac{p}{r'}_{L^p} (\int_G \abs{f(y)}^p\abs{g(y^{-1}x)}^r d\lambda(y) )^\frac{1}{q} (\int_G \abs{g(y^{-1}x)}^r d\lambda(y) )^\frac{1}{p'} \\
&= \norm{f}^\frac{p}{r'}_{L^p} (\int_G \abs{f(y)}^p\abs{g(y^{-1}x)}^r d\lambda(y) )^\frac{1}{q} (\int_G \abs{g(y^{-1})}^r d\lambda(y) )^\frac{1}{p'} \\
&= \norm{f}^\frac{p}{r'}_{L^p} (\int_G \abs{f(y)}^p\abs{g(y^{-1}x)}^r d\lambda(y) )^\frac{1}{q} \norm{g}^\frac{r}{p'}_{L^r} \\
&= (\int_G \abs{f(y)}^p\abs{g(y^{-1}x)}^r d\lambda(y) )^\frac{1}{q} \norm{g}^\frac{r}{p'}_{L^r} \norm{f}^\frac{p}{r'}_{L^p}
\end{align*}

Where, in the final calculating step, we have made use of the fact that we are integrating over everything and the abelian property of our topological group \footnote{Note however that this can also be proven without the abelian property, albeit with a few minor prerequisites.}.
We now once more take the $L^q$ norm of the inequality since everything is absolute and thus the inequality will be preserved. Notice we shall take the norm in $x$, respecting the parameter of our convolution whilst preserving the two other norms:

\begin{align*}
\norm{\abs{f} \ast \abs{g}}_{L^q} &\leq (\int_G \abs{(\int_G \abs{f(y)}^p\abs{g(y^{-1}x)}^r d\lambda(y) )^\frac{1}{q}}^q d\lambda(x))^\frac{1}{q} \norm{g}^\frac{r}{p'}_{L^r} \norm{f}^\frac{p}{r'}_{L^p} \\
&= (\int_G\int_G \abs{f(y)}^p\abs{g(y^{-1}x)}^r d\lambda(x)) d\lambda(y) )^\frac{1}{q} \norm{g}^\frac{r}{p'}_{L^r} \norm{f}^\frac{p}{r'}_{L^p} \\
&= (\int_G\int_G \abs{g(y^{-1}x)}^r d\lambda(x) \abs{f(y)}^p d\lambda(y)  )^\frac{1}{q} \norm{g}^\frac{r}{p'}_{L^r} \norm{f}^\frac{p}{r'}_{L^p} \\
&= (\int_G \abs{g(x)}^r d\lambda(x) \int_G \abs{f(y)}^p d\lambda(y)  )^\frac{1}{q} \norm{g}^\frac{r}{p'}_{L^r} \norm{f}^\frac{p}{r'}_{L^p} \\
&= \norm{g}^\frac{r}{q}_{L^r} \norm{f}^\frac{p}{q}_{L^p} \norm{g}^\frac{r}{p'}_{L^r} \norm{f}^\frac{p}{r'}_{L^p} \\
&= \norm{g}_{L^r} \norm{f}_{L^p} \\
&< \infty
\end{align*}  

As in Minkowski's Proof previously, we conclude $f \ast g$ exists $\lambda$-a.e. and satisfies the theorem.
\end{proof}

We are left wondering if there are similar results for weak $L_p$ spaces, fortunately Young has not left us hanging.

\begin{theorem}
Young's Inequality for Weak Type Spaces. Let $1\leq p < \infty$, $1<q$, and $r<\infty$ such that
\begin{equation*}
\frac{1}{q}+1 = \frac{1}{p}+\frac{1}{r}
\end{equation*} 
Then for all $f \in L^p$ and $g \in L^{r,\infty}$ $f\ast g$ exists $\lambda$-a.e. and the following inequality holds:
\begin{equation*}
\norm{f \ast g}_{L^{q,\infty}} \leq C_{p,q,r}\norm{g}_{L^{r,\infty}} \norm{f}_{L^{p}}
\end{equation*}
where $C_{p,q,r} > 0$ is a constant dependant only on $p$,$q$, and $r$.
\end{theorem}


\begin{proof}

Fix $M \in \mathbb{R} : M>0$. We shall pick a precise $M$ later in the proof when we know how we want it to look like. 
The proof of this theorem once more splits the terms into more manageable pieces, this time we shall work with $g$ in the form of:
\begin{enumerate}
\item $g_1 := g\chi_{\abs{g}\leq M}$
\item $g_2 := g\chi_{\abs{g} > M}$ 
\end{enumerate}

By application of Exercise 1.1.10(a) from the previous section we find:

\begin{enumerate}
\item $d_{g_1}(\alpha) = 
\begin{cases}
\displaystyle
0 & if \alpha \geq M \\
\displaystyle
d_g(\alpha) - d_g(M) & if \alpha < M
\end{cases}$
\item $d_{g_2}(\alpha) = 
\begin{cases}
\displaystyle
d_g(\alpha) & if \alpha > M \\
\displaystyle
d_g(M) & if \alpha \leq M
\end{cases}$
\end{enumerate}

Having chosen this parting of $g$ we quickly see that:

\begin{align*}
(f \ast g_1) + (f\ast g_2) &= f \ast (g_1 + g_2) \\
&= f \ast g
\end{align*}

And thus we call to mind Proposition 1.1.3 (3), and by substitution of the above equality we find
\begin{equation*}
d_{f \ast g}(\beta) \leq d_{f \ast g_1} (\frac{\beta}{2}) + d_{f \ast g_2} (\frac{\beta}{2})
\end{equation*}
must hold for all $\beta > 0$. We thus find we may bound $g_1$ and $g_2$ separately to come to a result. We begin by bounding $g_1$, and since $g_1$ is the cut below $M$ we see that $g_1 \in L^s$ as long as $s > r$ by the simple fact of $g \in L^{r,\infty}$. Additionally we assume $s < \infty$. Applying Proposition 1.1.4, we find

\begin{align*}
\norm{g_1}_{L^s}^s &= \int_G g_1(x)^s d\lambda(x)\\
&= s \int_0^\infty \alpha^{s-1} d_{g_1}(\alpha) d\alpha \\
&= s \int_0^M \alpha^{s-1} (d_{g}(\alpha)-d_{g}(M)) d\alpha \\
&= s \int_0^M \alpha^{s-1} \frac{\alpha^r}{\alpha^r} (d_{g}(\alpha)-d_{g}(M)) d\alpha \\
&= s \int_0^M \alpha^{s-1-r} \alpha^r (d_{g}(\alpha)-d_{g}(M)) d\alpha \\
&= s \int_0^M \alpha^{s-1-r} (\alpha^r d_{g}(\alpha)) d\alpha - s \int_0^M \alpha^{s-1} d_{g}(M) d\alpha \\
&\leq s \int_0^M \alpha^{s-1-r} \norm{g}^r_{L^{r,\infty}} d\alpha - s \int_0^M \alpha^{s-1} d_{g}(M) d\alpha \\
&= \frac{s}{s-r}M^{s-r} \norm{g}^r_{L^{r,\infty}} - M^s d_g(M)
\end{align*}
where we have applied our definition of $g_1$, removed the $0$ part, and made use of the fact that $\norm{g}_{L^{r,\infty}} \leq \norm{g}_{L^{r}}$.
In an almost analogous manner we can estimate $g_2 \in L^t$ for $t < r$ to find
\begin{align*}
\norm{g_2}_{L^t}^t &= \int_G g_2(x)^t d\lambda(x)\\
&= t \int_0^\infty \alpha^{t-1} d_{g_2}(\alpha) d\alpha \\
&= t \int_0^M \alpha^{t-1} d_{g}(M) d\alpha + t \int_M^\infty \alpha^{t-1} d_{g}(\alpha) d\alpha \\
&\leq M^t d_g(M) + t \int_M^\infty \alpha^{t-1-r} \norm{g}^r_{L^{r,\infty}} d\alpha \\
&\leq M^{t-r} \norm{g}^r_{L^{r,\infty}} + \frac{t}{r-t} M^{t-r} \norm{g}^r_{L^{r,\infty}}\\
&= (1+\frac{t}{r-1}) M^{t-r} \norm{g}^r_{L^{r,\infty}} \\
&= (\frac{r}{r-t}) M^{t-r} \norm{g}^r_{L^{r,\infty}}
\end{align*} 
where we have applied the same strategy as before.

As in Young's other proof, we examine the relationship between the exponents again and notice that $\frac{1}{r} = \frac{1}{p'} + \frac{1}{q} \implies 1 < r < p'$. In hindsight we shall let $t = 1$ and $s = p'$, satisfying the previously set requirements for the constants. We apply H\"older's to our estimate of $g_1$ to receive

\begin{align*}
\abs{ f \ast g_1)(x)} &\leq \norm{f}_{L^p}\norm{g_1}_{L^{p'}} \\
&\leq \norm{f}_{L^p} (\frac{p'}{p'-r}M^{p'-r} \norm{g}^r_{L^{r,\infty}})^\frac{1}{p'}
\end{align*}

When $p' = \infty$ this final term conveniently reduces to:

\begin{equation}
\abs{ f \ast g_1)(x)} \leq \norm{f}_{L^p}M
\end{equation}
 
We now choose $M$ such that if we substitute it into one of the two above final terms, in both cases we shall receive $\frac{\beta}{2}$. With some calculations we find that

\begin{equation}
M = \begin{cases}
\displaystyle
(\beta^{p'} 2^{-p'} rq^{-1} \norm{f}^{-p'}_{L^p} \norm{g}^{-r}_{L^{r,\infty}})^\frac{1}{p'-r} & \text{if } p' < \infty \\
\displaystyle
\frac{\beta}{2 \norm{f}_{L^1}} & \text{if } p' = \infty
\end{cases}
\end{equation}

meets our requirements without violating any previous assumptions. It follows from prior discoveries that when $M$ is such then
\begin{equation}
d_{f \ast g_1}(\frac{\beta}{2})= 0
\end{equation}

and thus we shall continue with $g_2$ to complete 
our proof. By applying Minkowski's inequality to our previous bounding of $g_2$ we have

\begin{align*}
\norm{f \ast g_2}_{L^p} &\leq \norm{f}_{L^p} \norm{g_2}_{L^1} \\
&\leq \norm{f}_{L^p} \frac{r}{r-1} M^{1-r} \norm{g}^r_{L^{r,\infty}}
\end{align*}

We now have everything required to bound everything:

\begin{align*}
d_{f \ast g}(\beta) &\leq d_{f \ast g_1}(\frac{\beta}{2}) + d_{f \ast g_2}(\frac{\beta}{2}) \\
&= d_{f \ast g_2}(\frac{\beta}{2} \\
&\leq (2 \norm{f \ast g_2}_{L^p} \beta^{-1}) ^ p \\
&\leq (2 (r \norm{f}_{L^p} M^{1-r} \norm{g}^r_{L^{r,\infty}} (r-1)^{-1}) \beta^{-1})\\
&= C^q_{p,q,r} \beta^{-q} \norm{f}^q_{L^p} \norm{g}^q_{L^{r,\infty}}
\end{align*}

Here we have substituted $M$ and applied Chebyshev's inequality for the second step. We see that the inequality in the theorem holds and that 
\begin{equation*}
C_{p,q,r} \to (r-1)^{-\frac{p}{q}} \text{ when } r \to 1.
\end{equation*}

concluding our proof.

\end{proof}

We must however make the reader aware of the failing edge cases of Young's Weak Inequality. It is also clear to see that when $r=q=\infty \wedge p=1$ the equality is trivial.

\begin{remark}
Let $G=\mathbb{R}$ with the additive structure and $r=1 \wedge 1\leq p =q \leq \infty$. We shall choose the functions $g=\frac{1}{\abs{x}}$ and $f=\chi_{[0,1]}$. It is easy to see that $g \in L^{1,\infty}$ and $f \in L^p$. Inspecting $f \ast g$ on the interval $[0,1]$ however yields that
\begin{align*}
(f \ast g)(x) &= \int_\mathbb{R} f(y)g(x-y) d\lambda(y)\\
&= \int_\mathbb{R} \chi_{[0,1]}(y) \frac{1}{\abs{x-y}} d\lambda(y) \\
&= \int_0^1 \frac{1}{\abs{x-y}} d\lambda(y) \\
&= \infty
\end{align*} 
clearly contradicting Young's Weak.
\end{remark}

\begin{remark}
Once more let $G=\mathbb{R}$. For exponents we take $q = \infty \wedge 1<r = p' < \infty.$ For functions:
\begin{equation*}
f = 
\begin{cases}
\displaystyle
\frac{1}{\abs{x}^\frac{1}{p} log\abs{x}} & if \abs{x} \geq 2 \\
\displaystyle
0 & else
\end{cases} 
\end{equation*}
and
\begin{equation*}
g = \abs{x}^\frac{-1}{r}
\end{equation*}
In a similar vein to the previous remark we find that $(f \ast g)(x) = \infty$ on the interval $[-1,1]$, contradicting Young's Weak.
\end{remark}



\section{Approximate Identities}

We are left wondering if there is an identity element under convolution, a function $g$ such that 
\begin{equation*}
\forall f \in L^1(G): g \ast f = f \ast g = f
\end{equation*}
In the next chapter the Dirac delta distribution, which full fills this quality, is introduced. However since this is not a function, we shall make do with the concept of approximating such a $g$ with a series of functions with appropriate characteristics.

\begin{definition}
An \textbf{approximate identity} (as $\varepsilon \rightarrow 0$) is a family of $L^1(G)$ functions $k_\varepsilon$ with the following three properties: 
\begin{enumerate}[label=(\roman*)]
\item There exists a constant $c > 0$ such that $\| k_\varepsilon \| _ {L^1(G)} \leqslant c$ for all $ \varepsilon > 0 $ . 
\item $ \int_G k_\varepsilon(x) d\lambda(x) = 1$ for all $\varepsilon > 0 $.
\item For any neighborhood $V$ of the identity element $e$ of the group G we have $ \int_{V^c} | k_\varepsilon(x) | d \lambda(x) \rightarrow 0  $ as $ \varepsilon \rightarrow 0 $ . 
\end{enumerate}
\end{definition}

\begin{theorem}
Any approximate identity has the following two properties:
\begin{enumerate}
\item $f \in L^p(G) \wedge 1 \leq p < \infty \implies \norm{k_\epsilon \ast f - f}_{L^p(G)} \to 0 $ as $ \epsilon \to 0$
\item $f \in L^\infty(G)$ uniformly continuous on $K \subset G \implies \norm{k_\epsilon \ast f - f}_{L^\infty(K)} \to 0$ as $\epsilon \to 0$ \\
Furthermore, if $f$ is bounded and continuous at $x \in G$ then
$ (k_\epsilon \ast )(x) \to f(x) $ as $\epsilon \to 0$ 
\end{enumerate}
\end{theorem}

\begin{proof}
Let us first prove the statement in finite $L^p$ spaces. We shall make use of the following equality:
\begin{align*}
\forall p \in \mathbb{N} : \abs{g(h^{-1}x) - g(x)}^p &\leq (\abs{g(h^{-1}x)}+\abs{g(x)})^p \\
&\leq (\esssup_x (\abs{g(h^{-1}x)} )+\esssup_x(\abs{g(x)}) )^p \\
&\leq (2\norm{g}_{L^\infty})^p
\end{align*}
Applying the Lebesgue Dominated Convergence theorem to the above restricted sequentially on the domain we find:
\begin{align*}
\int_G \abs{g(h^{-1}x)-g(x)}^p d\lambda(x) \to 0 \text{ as } h \to e
\end{align*}
Where $e$ is the neutral element of $G$. In $\mathbb{R}^n$ this is simply $0$. We can approximate any $g \in L^p(G)$ with a continuous function $f$ with compact support. Thus the property still holds:
\begin{align*}
\int_G \abs{f(h^{-1}x)-f(x)}^p d\lambda(x) \to 0 \text{ as } h \to e
\end{align*}
However we can now say that since $f$ is continuous,
\begin{equation}
\delta > 0 : \exists V(e): h \in V(e) \implies \int_G \abs{f(h^{-1}x) - f(x)}^p d\lambda(x) < (\frac{\delta}{2})^p ( \frac{1}{c})
\end{equation}
Where $c$ is from property (1) of our approximate identity and $V(e)$ is a neighborhood of $e$. We shall fix this neighborhood for later. We have picked the value on the right side for later convenience. As with most proofs in this area, we shall split the object of our analysis into two parts we can evaluate:
\begin{align*}
(k_\epsilon \ast f)(x) - f(x) &= \int_G (f(y^{-1}x)-f(x))k_\epsilon(y) d\lambda(y)
\end{align*}
We take $L^p$ norms on both sides, 
\begin{align*}
\norm{(k_\epsilon \ast f)(x) - f(x)}_{L^p(G)} &= \norm{\int_G (f(y^{-1}x)-f(x))k_\epsilon(y) d\lambda(y)}_{L^p(G)} \\
&= (\int_G \abs{\int_G (f(y^{-1}x)-f(x))k_\epsilon(y) d\lambda(y)}^p d\lambda(x) )^{\frac{1}{p}}\\
&\leq (\int_G \int_G \abs{f(y^{-1}x)-f(x)}^p \abs{k_e(y)}  d\lambda(y) d\lambda(x) )^{\frac{1}{p}} \\
&= (\int_G \int_V \abs{f(y^{-1}x)-f(x)}^p \abs{k_e(y)} d\lambda(y) d\lambda(x) 
+ \int_G \int_{V^c} \abs{f(y^{-1}x)-f(x)}^p \abs{k_e(y)} d\lambda(y) d\lambda(x) )^{\frac{1}{p}}
\end{align*}
Where the inequality originates from Jensen's inequality for integrals. Notice that we can now inspect on $V$ and $V^c$, we shall call the respective parts of the function $F_V$ and $F_{V^c}$ such that our last statement can be rewritten as $ ( F_V + F_{V_c} )^{\frac{1}{p}}$ for convenience. We now look at them individually. 

\begin{align*}
F_V &= \int_G \int_V \abs{f(y^{-1}x)-f(x)}^p \abs{k_e(y)} d\lambda(y) d\lambda(x)\\
&= \int_V \int_G \abs{f(y^{-1}x)-f(x)}^p \abs{k_e(y)} d\lambda(x) d\lambda(y)\\
&= \int_V \int_G \abs{f(y^{-1}x)-f(x)}^p  d\lambda(x) \abs{k_e(y)} d\lambda(y)\\
&\leq \int_V ( (\frac{\delta}{2})^p \frac{1}{c} ) \abs{k_e(y)} d\lambda(y)\\
&= ( (\frac{\delta}{2})^p \frac{1}{c} ) \int_V  \abs{k_e(y)} d\lambda(y)\\
&\leq ( (\frac{\delta}{2})^p \frac{1}{c} ) \int_G  \abs{k_e(y)} d\lambda(y)\\
&\leq ( (\frac{\delta}{2})^p \frac{1}{c} ) c\\
&= ( (\frac{\delta}{2})^p )\\
\end{align*}
Here we have used Fubini, substituted our previously discovered upper bound, and used property $(i)$ intrinsic of any approximate identity, as defined. We bound the other half now:

\begin{align*}
F_{V^p} &= \int_G \int_{V^c} \abs{f(y^{-1}x)-f(x)}^p \abs{k_\epsilon(y)} d\lambda(y) d\lambda(x) \\
&= \int_{V^c} \int_{G} \abs{f(y^{-1}x)-f(x)}^p \abs{k_\epsilon(y)} d\lambda(x) d\lambda(y) \\
&= \int_{V^c} \norm{f(y^{-1}x)-f(x)}^p_{L^p} \abs{k_\epsilon(y)} d\lambda(x) d\lambda(y) \\
&\leq \int_{V^c} (\norm{f}^p_{L^p}+\norm{f}^p_{L^p}) \abs{k_\epsilon(y)} d\lambda(x) d\lambda(y) \\
&= \int_{V^c} (2\norm{f}^p_{L^p}) \abs{k_\epsilon(y)} d\lambda(x) d\lambda(y) \\
&= (2\norm{f}^p_{L^p}) \int_{V^c} \abs{k_\epsilon(y)} d\lambda(x) d\lambda(y) \\
&\leq (2\norm{f}^p_{L^p}) \int_{V^c} \abs{k_\epsilon(y)} d\lambda(x) d\lambda(y) \\
&\leq ( (\frac{\delta}{2})^p )
\end{align*}
Were we have proceeded as before with the slight variation of applying Minkowski's, and having chosen a correct neighbourhood $V^c$. Putting our function back together we find:

\begin{align*}
(F_V + F_{V^c}) &\leq (\frac{\delta}{2})^p + (\frac{\delta}{2})^p ) ^{\frac{1}{p}}\\
&\leq (\frac{\delta}{2})+(\frac{\delta}{2}))\\
&= \delta\\
\end{align*}

Where the final inequality follows from the fact that $p \geq 1 \implies \frac{1}{p} \leq 1$ and Ex. 1.1.4 from the previous section. We have proven property $(1)$ of our theorem since the norm becomes arbitrarily small and thus converges to 0.

We shall now prove the case of $ p = \infty$. First we prepare some ingredients. We have required that f is bounded and uniformly continuous. Recall:

\begin{align*}
f \text{ uniformly continuous} \implies \delta > 0 : \forall  V(e): \abs{f(y^{-1}x)-f(x)} < \frac{\delta}{2c}
\end{align*}

Furthermore from property $(iii)$ of an approximate identity we have:

\begin{align*}
\exists \epsilon_0 > 0 : \forall \epsilon \in (0,\epsilon_0): \int_{V^c} \abs{k_\epsilon(y)} d\lambda(y) \leq \frac{\delta}{2(\norm{f}_{L^\infty}+1)}
\end{align*}

Once more we shall split our normed function:

\begin{align*}
\norm{k_\epsilon \ast f -f}_{L^\infty}(G) &= \sup_{x\in G} \abs{(k_\epsilon \ast f)(x)-f(x)} \\
&= \sup_{x\in G} \abs{\int_G f(y^{-1}x)-f(x) k_\epsilon(y) d\lambda(y)}\\
&= \sup_{x\in G} \abs{\int_V f(y^{-1}x)-f(x) k_\epsilon(y) d\lambda(y) + \int_{V^c} f(y^{-1}x)-f(x) k_\epsilon(y) d\lambda(y)}\\
&\leq \sup_{x\in G} \abs{\int_V f(y^{-1}x)-f(x) k_\epsilon(y) d\lambda(y)} + \abs{\int_{V^c} f(y^{-1}x)-f(x) k_\epsilon(y) d\lambda(y)} \\
&\leq \sup_{x\in G} \abs{\int_V f(y^{-1}x)-f(x) k_\epsilon(y) d\lambda(y)} + \sup_{x\in G} \abs{\int_{V^c} f(y^{-1}x)-f(x) k_\epsilon(y) d\lambda(y)} \\
&\leq \sup_{x\in G} \int_V \abs{f(y^{-1}x)-f(x)}\abs{ k_\epsilon(y) d\lambda(y)} + \sup_{x\in G} \int_{V^c}\abs{ f(y^{-1}x)-f(x)}\abs{ k_\epsilon(y) d\lambda(y)}\\
&\leq  \int_V \sup_{x\in G}\abs{f(y^{-1}x)-f(x)}\abs{ k_\epsilon(y) d\lambda(y)} +  \int_{V^c}\sup_{x\in G}\abs{ f(y^{-1}x)-f(x)}\abs{ k_\epsilon(y) d\lambda(y)}\\
&\leq  \int_V (\frac{\delta}{2c}) \abs{ k_\epsilon(y) d\lambda(y)} +  \int_{V^c}\sup_{x\in G}\abs{ f(y^{-1}x)-f(x)}\abs{ k_\epsilon(y) d\lambda(y)}\\
&\leq (\frac{\delta}{2c}) \int_V  \abs{ k_\epsilon(y) d\lambda(y)} +  \int_{V^c}\sup_{x\in G}\abs{ f(y^{-1}x)-f(x)}\abs{ k_\epsilon(y) d\lambda(y)}\\
&\leq (\frac{\delta}{2c}) \int_G  \abs{ k_\epsilon(y) d\lambda(y)} +  \int_{V^c}\sup_{x\in G}\abs{ f(y^{-1}x)-f(x)}\abs{ k_\epsilon(y) d\lambda(y)}\\
&\leq (\frac{\delta}{2c}) c +  \int_{V^c}\sup_{x\in G}\abs{ f(y^{-1}x)-f(x)}\abs{ k_\epsilon(y) d\lambda(y)}\\
&= (\frac{\delta}{2}) +  \int_{V^c}\sup_{x\in G}\abs{ f(y^{-1}x)-f(x)}\abs{ k_\epsilon(y) d\lambda(y)}\\
&\leq \frac{\delta}{2} + \frac{\delta}{2}
&\leq \delta
\end{align*}
Here we have normed our required equality, split the integral into neighbourhood and complement, used the triangle inequality, and applied previous definitions and the preparatory discoveries. Note that in the case of $\norm{f}^\infty = 0$ we have increased our denominator slightly as to avoid division by $0$.

This concludes the proof of both parts.
\end{proof}

\begin{theorem}
Let $k_\epsilon$ be an approximate identity with a slight variation, instead of satisfying property $(ii)$ is satisfies:

\begin{align*}
(ii)' :\Leftrightarrow a\in \mathbb{C} \wedge \forall \epsilon > 0 : \int_G k_\epsilon(x) d\lambda(x) = a 
\end{align*}

then similarly the following is true:

\begin{enumerate}
\item $f \in L^p(G) \wedge 1 \leq p < \infty \implies \norm{k_\epsilon \ast f - af}_{L^p(G)} \to 0 $ as $ \epsilon \to 0$
\item $f \in L^\infty(G)$ uniformly continuous on $K \subset G $such that 
\begin{align*}
\forall \delta>0: \exists V(e): sup_{x\in G}sup_{y \in V(e)}\abs{f(y^{-1}x)-f(x)}\leq \delta
\end{align*}
 then $ \norm{k_\epsilon \ast f - af}_{L^\infty(K)} \to 0$ as $\epsilon \to 0$  
\end{enumerate}

\end{theorem}

\begin{proof}
The proof is analogous to the previous proof, working with $a$ instead of $1$.\end{proof}

Let us inspect a typical case in order to help us visualize the behaviour of an approximate identity:

\begin{example}

A useful example of an approximate identity is the Poisson kernel on $\mathbb{R}$, defined as:

\begin{align*}
P(x) &:= (\pi(x^2+1))^{-1} \\
P_\epsilon(x) &:= \epsilon^{-1}P(\epsilon^{-1}x)
\end{align*}

Notice first a convenience:

\begin{align*}
\norm{P}_{L^1} &= \int_\mathbb{R} \abs{\frac{1}{ \pi (x^2+1)} } d\lambda(x) \\
&= \int_\mathbb{R} \frac{1}{ \pi (x^2+1)} d\lambda(x) \\
&= \int_\mathbb{R} \frac{1}{\epsilon(\pi (\frac{x}{\epsilon})^2+1)} d\lambda(x)\\
&= \int_\mathbb{R} \abs{ \frac{1}{\epsilon(\pi (\frac{x}{\epsilon})^2+1)}} d\lambda(x)\\
&= \norm{P_\epsilon}_{L^1}
\end{align*}

by a change of variables $x$ to $\frac{x}{\epsilon}$ and noticing that the function is positive. With this insight we can prove required property $(i)$ and $(ii)$:
\begin{align*}
\int_\mathbb{R} \frac{1}{\pi(x^2+1)} d\lambda(x) &= \int_{-\infty}^{+\infty} \frac{1}{\pi(x^2+1)} dx \\
&= \lim_{a\to +\infty} \int_{-a}^{+a} \frac{1}{\pi(x^2+1)} dx \\
&= \lim_{a\to +\infty} \frac{1}{\pi}( \arctan(x) \vert_{-a}^{+a}) \\
&= \lim_{a\to +\infty} \frac{1}{\pi} (\arctan(a) - \arctan(-a)) \\
&= \frac{1}{\pi} (\frac{\pi}{2} - (- \frac{\pi}{2})) \\
&= 1
\end{align*}

Notice that $(ii) \wedge (k_\epsilon \leq 0) \implies (i)$ on $\mathbb{R}$. To prove $(iii)$ we notice that $V^c(e)$ on $\mathbb{R}$ is simply the set $\cbr{ x \in \mathbb{R} : \abs{x} \geq \delta}$ for some $\delta > 0$. We integrate:

\begin{align*}
\int_{\cbr{ x \in \mathbb{R} : \abs{x} \geq \delta}} \abs{P_\epsilon} d\lambda(x) &= \int_{\cbr{ x \in \mathbb{R} : \abs{x} \geq \delta}} \abs{\frac{1}{\epsilon}\frac{1}{\pi((\frac{x}{\epsilon})^2+1)}} d\lambda(x) \\
&= \frac{1}{\pi} \int_{\cbr{ x \in \mathbb{R} : \abs{x} \geq \delta}} \abs{\frac{1}{\epsilon}\frac{1}{((\frac{x}{\epsilon})^2+1)}} d\lambda(x)\\
&= 1 - \frac{2}{\pi} \arctan(\frac{\delta}{\epsilon})
\end{align*}

Which goes to $0$ as $\epsilon \to 0$ since $\arctan(\frac{\delta}{\epsilon}) \to \frac{\pi}{2}$. Thus we have shown that the Poisson kernel is an approximate identity. Figure 1 depicts this kernel.

\begin{figure}[h!bt]
\centering
\includegraphics[scale=0.7]{matlab/poissonkernel}
\caption{Poisson Kernel}
\label{fejer}
\end{figure}

\end{example}

\begin{remark}
It is possible to state a more general form on $\mathbb{R}^n$ for an approximate identity which contains the Poisson kernel. Let $\int_{\mathbb{R}^n} k(x) = 1$ and define $k_\epsilon(x) = \epsilon^{-n}k(\epsilon^{-1}x)$. We see $k_\epsilon$ is an approximate identity by a similar notion to the proof of the Poisson kernel.
\end{remark}

\begin{example}
Another example of an approximate identity is the Fejer Kernel defined as follows:
\begin{equation*}
F_N(t) := \frac{1}{N+1}(\frac{sin(\pi(N+1)t)}{sin(\pi t)})^2
\end{equation*}
on the circle group $T^1$. It is graphed beautifully in Figure 2.

\begin{figure}[h!bt]
\centering
\includegraphics[scale=0.7]{matlab/fejerkernel}
\caption{Fejer Kernel}
\label{fejer}
\end{figure}

\end{example}

\begin{appendix}
\section{Poisson Kernel}
The matlab code for graphing the Poisson kernel:

\inputminted{matlab}{matlab/aproIdInR.m}

\section{Fejer Kernel}
The matlab code for graphing the Fejer kernel:
\inputminted{matlab}{matlab/fejerkernel.m}

\end{appendix}

\end{document}